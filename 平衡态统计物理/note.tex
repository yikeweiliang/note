\documentclass[a4paper, 10pt, openany]{book}%标注了文档类型和字号大小
\usepackage{geometry}

\geometry{left=3cm,right=3cm,top=3cm,bottom=3cm}
\usepackage[T1]{fontenc}
\usepackage{titlesec}
\usepackage{hyperref}
\usepackage{url} % 用于更好地格式化URL

\titleformat{\section}[block]{\normalfont}{\thesection}{1em}{}
\titleformat{\subsection}[block]{\normalfont}{\thesubsection}{1em}{}
\titleformat{\chapter}[display]
  {\normalfont\huge} % 设置章节标题的字体为正常字体(非加粗)
  {Chapter \thechapter}{1em}{} % 保留 "Chapter" 和 章节编号

\usepackage{xeCJK}
\usepackage{footmisc}
\usepackage[UTF8]{ctex}
\usepackage{amsmath}
\usepackage{subfigure}
\usepackage[graphicx]{realboxes}



\setCJKmainfont{FangSong}%中文设置为仿宋

\setmainfont{Palatino Linotype}%英文设置为palatino linotype

\begin{document}
  \title{ \heiti 平衡态统计物理笔记}
  \author{亦可}
  \maketitle
  \tableofcontents


  \newpage

  \chapter{写在前面/Foreword}
  \chapter{热学回顾/Thermodynamics}

  热学理论是唯象的,唯象的观点没有所谓对体系微观细节的认知,热力学定律基本来自实验和日常规律的总结。

  统计物理和热力学研究的核心问题,总是假定体系处于平衡状态。它们没有办法回答“如何达到平衡态”的问题。

\section{热平衡状态与第零定律}

\subsection{第零定律}
一般当我们提及平衡状态,我们指称的是,一个体系的某些性质随着观察的时间的推移不发生变化。

这是一个很值得推敲的说法。什么样的性质不随时间变化?如果我们试想一箱子气体,随着时间的推移,我们持续跟踪箱子中一个确定的粒子的性质,该粒子的动量(或者位置)当然是随时变化的,即使整个系统已经达到了所谓平衡。也就是说,热力学意义上的平衡是动态的。因此,我们所谓的不随时间变化,实际上已经隐含了宏观测量的描述:

\begin{itemize}
\item 测量的特征时间要比微观的运动时间长的多。

\item 我们测量的物理量一定是粗略的,缓慢的。
\end{itemize}

对于一个宏观系统,其自由度当然是很多很多的,要完整地、详尽地描述它需要极多的自由度。因此在我们所谓意义上的测量时,我们是提取了体系的一个特征来进行测量,即将一个超高维的相空间简化为了一些简单的热力学变量来进行测量。这些热力学变量需要符合前述“缓慢的、粗略的”定义。

有些热力学变量很直观,例如压强(描述了气体的力学特征),体积(描述了气体的几何特征)等。但这些特征并非热力学体系独有。那么,什么变量是跟“热”相关的特征量呢?我们当然已经知道这就是温度。

\vspace{10pt}

{第零定律:考虑三个系统A,B与C。A与C热平衡,且B与C热平衡,则可以推出A与B热平衡。}

\vspace{10pt}

\subsection{温度与物态方程}

第零定律如何说明了温度的存在?(注意,我们并没有要求ABC三个系统的相态。不过为了简单进行说明,我们可以考虑三个系统均为气体的情况。)


A与C热平衡,说明存在一个函数$f_{ac}$,满足方程(2.1)。
\begin{equation}
f_{AC}(p_A,V_A,p_C,V_C)=0
\end{equation}

同理,B与C热平衡,说明存在:
\begin{equation}
f_{BC}(p_B,V_B,p_C,V_C)=0
\end{equation}

改写一下,可以写为(2.3)的形式.
\begin{equation}
V_C=F_{AC}(p_A,V_A,p_C)=F_{BC}(p_B,V_B,p_C)
\end{equation}

第零定律表明,如果有上式成立,则应该有:
\begin{equation}
  f_{AB}(p_A,V_A,p_B,V_B)=0
  \end{equation}

即上推下。因此(2.3)式必可以约去$p_C$,因此有:
\begin{equation}
  g_1(p_A,V_A)=g_2(p_B,V_B)
  \end{equation}

  此时$g$既是温度函数,也是物态方程。
\begin{itemize}

\item eg1.理想气体的物态方程:$pV=Nk_BT$,其中$N$为粒子数。

\item eg2.范德瓦尔斯气体的物态方程:
\begin{equation}\begin{cases}(p+\frac{a}{v^2})(v-b)=k_BT \\ v=\frac{V}{N}\end{cases}\end{equation}

\end{itemize}

把理想气体\footnote{关于理想气体。}作为测温物质(因为其状态方程简单)(即C系统),可以定义温标。

\begin{equation}T=273.16K\frac{(pV)_{input}}{(pV)_{triple}}\end{equation}

其中$K$表示开尔文,是温度的单位。
\section{第一定律}

\subsection{第一定律的各种表述}
在探讨第一定律之前,我们需要先讨论几个有关对于“器壁”概念的描述并明确它们的定义。
\begin{itemize}
  \item “绝热”-系统与系统之间没有热交换。
 \item  “透热”-系统之间存在热交换。
 \item “孤立”-外界不能对系统做任何事。

\end{itemize}

\vspace{10pt}

第一定律:一个绝热的系统(改变状态的方式只有做功),从不同的做功路径,使系统从同初态到同末态,需要的功相同。

\vspace{10pt}

第一定律说明存在内能这个物理量(因为稳定的做功说明存在稳定的能量差),内能的定义通过式(2.8)完成。
\begin{equation}
W=U(p_2,V_2)-U(p_1,V_1)
\end{equation}

如果器壁变成透热的,则上式不成立。差距来自通过器壁的热交换,于是我们定义热量交换

\begin{equation}
  Q=U_2-U_1-W
  \end{equation}

更普遍更量化的第一定律就可以表示为下面的式(2.10):
\begin{equation}\Delta U=W+Q\end{equation}

也有很多时候,我们希望对过程进行细分化,这时候,我们需要用到第一定律的无穷小形式(2.11)
请注意这里$\text{\dj}$符号的使用,这表示该量与路径相关。


\begin{equation}\mathrm{d}U=\text{\dj}W+\text{\dj}Q\end{equation}

我们希望给$\text{\dj}W$和$\text{\dj}Q$一个用系统参量表示的表达式。出于这个目的,我们来讨论关于$W$的过程量的表达式,为此,我们需要引入准静态过程的概念。

准静态过程,指的是一个过程slow enough以至于在系统变化的任意时刻始终保持平衡状态。
在准静态过程下,做功的表达式如式:

\begin{equation}\text{\dj}W=-p\mathrm{d}v\end{equation}

更一般的情况下,有

\begin{equation}\text{\dj}W=\sum_{i}\textbf{F}_i\cdot \mathrm{d}\textbf{x}_i\end{equation}

其中,$\textbf{F}$是强度量,$\textbf{x}$为广延量(随体系的尺寸改变而改变)。


\subsection{热容}

定义热容:体系升高1开尔文所需要的热量。

\begin{equation}C=\frac{\text{\dj}Q}{\mathrm{d}T}\end{equation}

这里有两点值得注意。第一,热容是一个实验可测的量,许多物理问题都源于此。第二,该表达式中,吸热是依赖过程的量,所以热容也是依赖过程的。常用的热容有$C_v$和$C_p$分别为定容热容和定压热容。

\begin{equation}C_v=\left(\frac{\partial U}{\partial T}\right)_v\end{equation}

\begin{equation}C_p=\frac{\mathrm{d}U+p\mathrm{d}V}{\mathrm{d}T}_p=\left(\frac{\partial U}{\partial T}\right)_p+p\left(\frac{\partial V}{\partial T}\right)_p\end{equation}

理想气体的绝热自由膨胀:该实验的结果表明,膨胀后的气体体积和压强都发生了变化,但是温度却没有变化,这说明理想气体的内能是温度的一元函数。可以得到
\begin{equation}U_{ig}=U_{ig}(T)\end{equation}


因此对于理想气体,有式(2.18)成立。

\begin{equation}C_p-C_v=p\left(\frac{\partial V}{\partial T}\right)_p=Nk_B\end{equation}

\section{第二定律}

\subsection{热机 }
对第二定律的探索来源于热机。围绕热机效率的研究催生了第二定律的表述。一般的,热机由高温热源+热机主体(向外做功)+向低温处放热的一整套循环系统组成。注意,高温热源并不一定要具有特定温度。循环中所有吸热的点都被认为处在高温热源。

且,尤其需要注意的一点是,由于是准静态过程,因此在吸放热时,系统的温度与热源或散热区一定要保持相同。\footnote{这里,我们需要重申一下可逆过程和准静态过程的关。一个没有耗散的准静态过程就是可逆过程。可逆过程一定是准静态过程。}

热机的主要功能是吸收热源的热量,并对外做功,当然,过程中热机还会放出一部分热量。对于此,我们总是希望热机尽可能多地把吸热转化为做功。也就是,如果我们定义热机效率:

\begin{equation}
  \eta =\frac{W}{Q_{in}}
\end{equation}

我们总是希望热机效率尽可能地高。显然,由于第一定律,我们知道:

\begin{equation}
\eta\leq1
\end{equation}

经过大量的实验观察,热力学第二定律终于得到总结。在经典的教材中,第二定律的克劳修斯表述和开尔文表述总是被给出。它们分别是:

\begin{itemize}
\item 克劳修斯表述:不可能将热从低温物体传递至高温物体,而不引起任何其他变化。

\item 开尔文表述:不可能存在这样一个热机,它从单一热源吸热,并把吸收的热量完全转化为做功。
\end{itemize}

\subsection{卡诺定理与热力学温标}
让我们继续热机的话题。对于此,卡诺给出的关于热机效率的研究结论是,在所有只在高温$T_1$处吸热,只在低温$T_2$处放热的热机中,可逆热机的效率最高。
我们在这里对热机加了比较严格的限制。一般的,不满足可逆过程的热机自不必说,即使是满足可逆过程的热机,也不一定必须在单一温度的热源吸热或者放热。(当然,为了保证可逆,它们在吸热过程中必须时刻保持和热源温度一致)




\subsection{熵}

















\end{document}