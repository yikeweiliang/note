\documentclass[a4paper, 10pt, openany]{book}%标注了文档类型和字号大小
\usepackage{geometry}

\geometry{left=3cm,right=3cm,top=3cm,bottom=3cm}
\usepackage[T1]{fontenc}
\usepackage{titlesec}
\usepackage{hyperref}
\usepackage{url} % 用于更好地格式化URL

\titleformat{\section}[block]{\normalfont}{\thesection}{1em}{}
\titleformat{\subsection}[block]{\normalfont}{\thesubsection}{1em}{}
\titleformat{\chapter}[display]
  {\normalfont\huge} % 设置章节标题的字体为正常字体(非加粗)
  {Chapter \thechapter}{1em}{} % 保留 "Chapter" 和 章节编号

\usepackage{xeCJK}
\usepackage{footmisc}
\usepackage[UTF8]{ctex}
\usepackage{amsmath}
\usepackage{subfigure}
\usepackage[graphicx]{realboxes}



\setCJKmainfont{FangSong}%中文设置为仿宋

\setmainfont{Palatino Linotype}%英文设置为palatino linotype

\begin{document}
  \title{ \heiti 平衡态统计物理笔记}
  \author{亦可}
  \maketitle
  \tableofcontents


  \newpage

  \chapter{写在前面/Foreword}
  \chapter{热学回顾/Thermodynamics}

  热学是唯象的,唯象的观点没有所谓对体系微观细节的认知,热力学定律基本来自实验和日常规律的总结。

  统计物理和热力学研究的核心问题,总是假定体系处于平衡状态。它们没有办法回答“如何达到平衡态”的问题。

\section{热平衡状态}
平衡状态:体系的性质随着观察的时间的推移不发生变化。

这是一个很tricky的说法。什么样的性质不随时间变化?如果我们试想一箱子气体,随着时间的推移,箱子中某一粒子的动量(或者位置)当然是随时变化的。因此,我们所谓的不随时间变化,实际上已经隐含了宏观测量的描述。测量的特征时间要比微观的运动时间长的多。我们测量的物理量一定是粗略的,缓慢的。

对于一个宏观系统,其自由度当然是很多很多的,要完整地描述它需要极多的自由度。因此在我们所谓意义上的测量时,我们是提取了体系的一个特征来进行测量,即将一个高维的相空间简化为了一些简单的热力学变量来进行测量。这些热力学变量即是符合前述“缓慢的、粗略的”定义。

有些热力学变量很直观,例如压强(描述了气体的力学特征),体积(描述了气体的几何特征)等。但这些特征并非热力学体系独有。那么,什么变量是跟“热”相关的特征量呢?我们当然已经知道这就是温度。

\subsection{第零定律}

考虑三个系统A,B与C。A与C热平衡,且B与C热平衡,则可以推出A与B热平衡。这就是第零定律。

注意,我们并没有要求ABC三个系统的相态。不过为了简单起见,我们可以考虑三个系统均为气体。

第零定律如何说明了温度的存在?

A与C热平衡,说明存在一个函数$f_{ac}$,使得$f_{AC}(p_A,V_A,p_C,V_C)=0$,同理存在$f_{BC}(p_B,V_B,p_C,V_C)=0$

改写一下,可以写为:$V_C=F_{AC}(p_A,V_A,p_C)=F_{BC}(p_B,V_B,p_C)$

第零定律表明,如果有上式成立,则有$f_{AB}(p_A,V_A,p_B,V_B)=0$,即上推下。因此上式必可以约去$p_C$,因此有$g_1(p_A,V_A)=g_2(p_B,V_B)$,此时$g$即温度函数。

注意此时的$g$即物态方程。

eg1.理想气体的物态方程:$pV=Nk_BT$,其中$N$为粒子数。

eg2.范德瓦尔斯气体的物态方程:
$$(p+\frac{a}{v^2})(v-b)=k_BT$$,其中$v=\frac{V}{N}$

把理想气体作为测温物质(因为其状态方程简单)(即C系统),可以定义温标。

$$T=273.16\frac{pV_{input}}{pV_{triple}}$$
\subsection{第一定律}
关于Wall,“绝热”:系统与系统之间没有热交换。“透热”:系统之间存在热交换。“孤立”:外界不能对系统做任何事。item

第一定律:一个绝热的系统(改变状态的方式只有做功),从不同的做功路径,使系统从同初态到同末态,需要的功相同。

第一定律说明存在内能这个物理量(因为稳定的做功说明存在稳定的能量差),定义内能为$W=U(p_2,V_2)-U(p_1,V_1)$

如果wall变成透热的,则上式不成立。差在哪里?定义热量交换$Q=U_2-U_1-W$

即$\Delta U=W+Q$

无穷小形式$dU=\text{\dj}W+\text{\dj}Q$,这与路径相关。

我们希望给$\text{\dj}W$和$\text{\dj}Q$一个用系统参量表示的表达式。我们来讨论关于$W$的过程量的表达式,为此,我们需要引入准静态过程的概念。

准静态过程:slow enough以至于在系统变化的任意时刻始终保持平衡状态。

在准静态过程下,有$\text{\dj}W=-pdv$

更一般的情况下,有$$\text{\dj}W=\sum_{i=0}^nF_i\cdot dx_i$$

其中,$F$是强度量,$x$为广延量(随体系的尺寸改变而改变)。

定义热容:体系升高一度所需要的热量。

$C=\frac{\text{\dj}Q}{dT}$

注意热容是一个实验可测的量,许多物理问题都源于此。

注意其中吸热是依赖过程的量,所以热容也是依赖过程的。常用的热容有$C_v$和$C_p$,记录两个表达式

$C_v=\frac{\partial U}{\partial T}_v$

$C_p=\frac{dU+pdV}{dT}_p=\frac{\partial U}{\partial T}_p+p\frac{\partial V}{\partial T}_p$

理想气体的绝热自由膨胀:该实验可以得到$U_{ig}=U_{ig}(T)$

因此理想气体有$C_p-C_v=p\frac{\partial V}{\partial T}_p=Nk_B$

\subsection{第二定律}
对第二定律的探索来源于热机。围绕热机效率的研究催生了第二定律的表述。

热机:高温热源+热机主体(向外做功)+向低温处放热

注意,高温热源并不一定要具有特定温度。循环中所有吸热的点都被认为处在高温热源。
















\end{document}