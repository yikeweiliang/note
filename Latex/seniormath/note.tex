\documentclass[a4paper, 10pt]{article}
\usepackage{geometry}
\geometry{left=3cm,right=3cm,top=3cm,bottom=3cm}
\usepackage{footmisc}
\usepackage[UTF8]{ctex}
\usepackage{amsmath}
\usepackage{subfigure}
\usepackage[graphicx]{realboxes}
\setlength{\parindent}{0pt} 
\begin{document}

According to the passage, there are several methods to stop the ant from spreading farther and ruiningthe ecosystem.However, the speaker raised oppositepoints about each of such methods and given herexplanations.

Firstly, it can seems be solved by using traps tocontaining a low consentration of chemicals that aretoxic to the ants. Nevertheless, the speaker pointedout that the chemicals used to kill the invasive antsmay also be harmful to the ants we want to protect, resulting in unintended damages, so it is not apracticle methods.

Besides,it appears reasonable to cut down broad-leaved trees to swipe the habitat of the ants. But thespeaker raised that there are 3 different habitats forants which not only involve the forests but also theplains.Cutting down the trees creates a new habitatfor them. So cut down the trees makes nocontribution to solve the problem.

Finally, it has been noted that ant populations are significantly denser near areas with human activity.The speaker acknowledge that it may slow down thepath of ant spreading by reduction of humanactivities, but such a solution would cause aneconomic problem. The tourists and scientists nearhere are the main sourses of local people's incomelt will incur a dramatically decrease on there lifelevel if we adopt the action to withdraw thescientists and forbid the tourists coming to here. Soas a result, it is also not a proper solution for thisproblem of local ecosystem.
\end{document}