\documentclass[a4paper, 10pt]{article}
\usepackage{geometry}
\geometry{left=3cm,right=3cm,top=3cm,bottom=3cm}
\usepackage{footmisc}
\usepackage[UTF8]{ctex}
\usepackage{amsmath}
\usepackage{subfigure}
\usepackage[graphicx]{realboxes}
\setlength{\parindent}{0pt} 
\begin{document}
\title{{ \textbf {part of reading}}}
  \author{yike}
  \date{}
  
  \maketitle
\section{25.1.18}
\subsection{note of class}
并列的提示词:

in addition to/additionally/then/and/plus/not only but also/\dots

笔记:

1.重点逻辑后的内容(逻辑符号)

2.名词>动词>形容词

3.不重复记

4.中英文哪个反应快写哪个

5.记辅音字母

6.画图

7.常见词对应符号

\section{25.1.20}
\subsection{note of class}

分段:考虑分段方式:1.提问(问题引入)2.过渡词(ok,so,now,lets,today)3.并列4.停顿 但不是学术名词后5.总结句

重听题:语境、语气、语义

假设选项是正确的,预测下文的发展,如果不是这么发展的 就不对

选项中的含义是否能够用这个语气表达,不能则不对

crop庄稼


\section{2.4}

主旨:全文主旨+段落主旨

分段方式:提问、过渡词、并列、对比

段落重点细节:并列(also,another)举例,类比(打比方,like,similar)举例和类比经常出修辞目的题why ,mention即问主旨

对比,比较

排除法(段间排除法:无:与主旨无关的选项,反:与段落态度或全文态度相反的选项,拼:拼凑跨段信息的选项)






\end{document}