\documentclass[a4paper, 10pt]{article}
\usepackage{geometry}
\geometry{left=3cm,right=3cm,top=3cm,bottom=3cm}
\usepackage{footmisc}
\usepackage[UTF8]{ctex}
\usepackage{amsmath}
\usepackage{subfigure}
\usepackage[graphicx]{realboxes}
\setlength{\parindent}{0pt} 
\begin{document}
\title{{ \textbf {homework day1}}}
  \author{yike}
  \date{}
  
  \maketitle

I 'm on favor of Andrew that the outdoor classes once a month is beneficial. 

Firstly, it provides a more efficient method to understand the world. Specifically speaking, outdoor classes remove the textbooks which acts as a medium between reality and children and give them a chance to perceive the world directly. For instance, the description of royal gardens would not be only sentences but also the actual breeze through the trees and waves on the surface of the lake when holding an outdoor class in the Summer Palace.   

Besides, it could balance the learning and relaxing of students. Considering the huge pressure undertaken by teenagers, it seems necessary to create an easy and cheerful atmosphere through some change from traditional classes. Students could enjoy themselves in such a relaxing environment while learning, painting a different color to their school life.



\end{document}